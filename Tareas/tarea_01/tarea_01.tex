\documentclass[12pt,letterpaper]{article}
\usepackage[utf8]{inputenc}
\usepackage[spanish]{babel}
\usepackage{amsmath}
\usepackage[hidelinks]{hyperref}
\usepackage{latexsym}
\usepackage{float}
\usepackage{subfigure}
\usepackage{array} %tabular
\usepackage{fancyhdr}
\usepackage{amsfonts}
\usepackage{amssymb}
\usepackage{natbib}
\usepackage{graphicx}
\usepackage{enumerate} % enumerados
\usepackage{cite}
\usepackage{wrapfig}
\usepackage{afterpage}
\setcitestyle{square}
\usepackage[export]{adjustbox}
\usepackage[left=2cm,right=2cm,top=2cm,bottom=2cm]{geometry}
\usepackage{afterpage}
\usepackage{setspace}
\spanishdecimal{.}


\makeatletter
\long\def\@makecaption#1#2{%
\vskip\abovecaptionskip
\sbox\@tempboxa{#1. #2}%
\ifdim \wd\@tempboxa >\hsize
#1. #2\par
\else
\global \@minipagefalse
\hb@xt@\hsize{\hfil\box\@tempboxa\hfil}%
\fi
\vskip\belowcaptionskip}
\makeatother


\title{Análisis Estadístico con Python 2024 \\ Tarea 01. Tipos de datos}
\author{Marc, M.}

\begin{document}

\maketitle

\spacing{1.5}



\begin{enumerate}
  \item El siguiente conjunto es una muestra o una población? \\ 
   \guillemotleft las edades de los miembros del gabinete actual de Japón\guillemotright . \\  
   Sol. \emph{población}
   
  \item El siguiente conjunto es una muestra o una población? \\ 
  \guillemotleft el salario de un grupo aleatoriamente seleccionado de 100 ciudadanos de Londres\guillemotright . \\ 
   Sol. \emph{muestra} 
   
   \item El siguiente conjunto es una muestra o una población? \\ 
  \guillemotleft los diámetros de los planetas de nuestro sistema solar\guillemotright . \\ 
   Sol. \emph{población} 
   
   \item El siguiente conjunto es una muestra o una población? \\ 
  \guillemotleft Enviamos un formulario de encuesta en línea a \textbf{todos} los alumnos de \textbf{SciData} para estimar el nivel de satisfacción. La tasa de respuesta fue del 11.4 \%. Al utilizar los datos de la encuesta devueltos, creamos un conjunto de datos de satisfacción del cliente\guillemotright . \\ 
   Sol. \emph{muestra}
   
   \item La variable que representa la temperatura máxima de cada ciudad de México es un tipo de dato \emph{cuantitativo} o \emph{cualitativo}? \\ 
   Sol. \emph{cuantitativo}
   
   \item La variable que representa tu número entero favorito es \emph{cuantitativa} o \emph{cualitativa}? \\ 
   Sol. \emph{cuantitativa}
   
   \item La variable que representa a la ciudad en que vive cada inscrito a este curso es \emph{cuantitativa} o \emph{cualitativa}? \\ 
   Sol. \emph{cualitativa} 
   
   \item Una variable que representa el Producto Interno Bruto de cada país en dólares estadounidenses es \emph{cuantitativa} o \emph{cualitativa}? \\ 
   Sol. \emph{cuantitativa} 
   
   \item El total de sillas en un salón de clases es un dato \emph{discreto} o \emph{continuo}? \\ 
   Sol. \emph{discreto} 
   
   \item El área en $m^{2}$ de un salón de clases es un dato \emph{discreto} o \emph{continuo}? \\ 
   Sol. \emph{continuo} 
   
   \item La variable que representa la motivación de un empleado en una escala del 1 al 5 es \emph{discreta} o \emph{continua}? \\ 
   Sol. \emph{continua}
   
   \item Una variable que representa el nivel de motivación laboral promedio de cada equipo medida para cada miembro del equipo mediante una escala de 5 puntos es \emph{discreta} o \emph{continua}? \\ 
   Sol. \emph{continua}
   
   \item El código postal es una variable \emph{nominal}, \emph{ordinal}, \emph{de intervalo} o \emph{de cociente}? \\ 
   Sol. \emph{nominal}
   
   \item El número en que se encuentra una universidad dentro de un rango de clasificación es una variable \emph{nominal}, \emph{ordinal}, \emph{de intervalo} o \emph{de cociente}? \\ 
   Sol. \emph{ordinal}
   
   \item El número en que se encuentra una universidad dentro de un rango de clasificación es una variable \emph{nominal}, \emph{ordinal}, \emph{de intervalo} o \emph{de cociente}? \\ 
   Sol. \emph{ordinal}
   
   \item La variable \guillemotleft altura de una montaña\guillemotright , es una variable \emph{nominal}, \emph{ordinal}, \emph{de intervalo} o \emph{de cociente}? \\ 
   Sol. \emph{cociente}
   
  %\item[!] A point to exclaim something!
  %\item[$\blacksquare$] Make the point fair and square.
  %\item[NOTE] This entry has no bullet
  %\item[] A blank label?
\end{enumerate}

\end{document}